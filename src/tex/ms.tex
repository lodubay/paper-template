% Define document class
\documentclass[twocolumn,twocolappendix]{aastex631}

\usepackage{graphicx}
\usepackage{xcolor}

% user-defined commands
\newcommand{\todo}[1]{{\color{red}#1}}
\newcommand{\osuaffil}{Department of Astronomy, The Ohio State University, 140 W. 18th Ave, Columbus OH 43210, USA}
\newcommand{\ccappaffil}{Center for Cosmology and AstroParticle Physics, The Ohio State University, 191 W. Woodruff Ave., Columbus OH 43210, USA}

\shorttitle{}
\shortauthors{}

\begin{document}

% Title
\title{}

% Author list
\author[0000-0003-3781-0747]{Liam O.\ Dubay}
\affiliation{\osuaffil}
\affiliation{\ccappaffil}

\correspondingauthor{}
\email{}

\begin{abstract}
    Abstract.
\end{abstract}

\section{Introduction}
\label{sec:introduction}

\section*{Acknowledgements}

% From the Center for Belonging and Social Change, https://cbsc.osu.edu/about-us/land-acknowledgement
We would like to acknowledge the land that The Ohio State University occupies is the ancestral and contemporary territory of the Shawnee, Potawatomi, Delaware, Miami, Peoria, Seneca, Wyandotte, Ojibwe and many other Indigenous peoples. Specifically, the university resides on land ceded in the 1795 Treaty of Greeneville and the forced removal of tribes through the Indian Removal Act of 1830. As a land grant institution, we want to honor the resiliency of these tribal nations and recognize the historical contexts that has and continues to affect the Indigenous peoples of this land.

\software{}

\appendix

\section{Reproducibility}
\label{app:reproducibility}

\bibliography{references}

\end{document}
